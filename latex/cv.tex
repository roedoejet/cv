\documentclass[12pt]{letter}
    \usepackage[LabelsAligned, NoDate]{currvita}
    \usepackage{tipa}
    \usepackage[margin=1.25in]{geometry}
    \date{}
    \pagenumbering{gobble}
    \begin{document}
        \begin{cv}{ Aidan Hart Ray Pine  \space - \space   Curriculum Vitae}
        \vspace{1mm}
      \begin{cvlist}{Education}
                        
                    \item[Sept 2012 - May 2016] \textbf{University of British Columbia}
                \newline B.A. Hon. Linguistics
                \newline Minor in First Nations Languages
                \newline 4.21 GPA (90.2\% Average)
        
                        
                    \item[Sept 2020 - Sept 2021] \textbf{University of Edinburgh}
                \newline M.Sc. Speech \& Language Processing
                \newline 1st Class Honours (80\% Coursework Average, 90\% Dissertation)
                \newline 
        
                    \end{cvlist}
        
        \begin{cvlist}{Publications}
                                                                        \item[2024a] Wells, D., Blanco, A., Valentini, C., Cooper, E., Pine, A., Yamagishi, J., Richmond, K.  Experimental evaluation of MOS, AB and BWS listening test designs. In proceedings of \textit{ Interspeech 2024 }.  ISCA.  
                                                                                \item[2024b] Nishihara, M., Wells, D., Richmond, K., Pine, A.  Low-dimensional Style Token Control for Hyperarticulated Speech Synthesis. In proceedings of \textit{ Interspeech 2024 }.  ISCA.  
                                                                                \item[2024c] Gong, C., Cooper, E., Wang, X., Qiang, C., Geng, M., Wells, D., Wang, L., Dang, J., Tessier, M., Pine, A., Richmond, K., Yamagishi, J.  An Initial Investigation of Language Adaptation for TTS Systems under Low-resource Scenarios. In proceedings of \textit{ Interspeech 2024 }.  ISCA.  
                                                                                \item[2024d] Littell, P.  Gramble: A Tabular Programming Language for Collaborative Linguistic Modeling. In proceedings of \textit{ LREC-COLING 2024 }.  ACL.  
                                                                                                            \item[2023a] Pine, A., Huggins-Daines, D., Joanis, E., Littell, P., Tessier, M., Torkornoo, D., Knowles, R., Kuhn, R., Lothian, D.  ReadAlong Studio Web Interface for Digital Interactive Storytelling. In proceedings of \textit{ 8th Workshop on Innovative Use of NLP for Building Educational Applications }.  ACL 2023.  
                                                                                                            \item[2022a] Pine, A., Wells, D., Brinklow, N., Littell, P., Richmond, K.  Requirements and motivations of low-resource speech synthesis for language revitalization. In proceedings of \textit{ 60th Annual Meeting of the Association for Computational Linguistics }.  ACL 2022.  
                                                                                \item[2022b] Pine, A., Littell, P., Joanis, E., Huggins-Daines, D., Cox, C., Davis, F., Santos, E., Srikanth, S., Torkornoo, D., Yu, S.  Gᵢ2Pᵢ: Rule-based, index-preserving grapheme-to-phoneme transformations. In proceedings of \textit{ 5th Workshop on the Use of Computational Methods in the Study of Endangered Languages }.  ComputEl-5.  
                                                                                \item[2022c] Littell, P., Joanis, E., Pine, A., Tessier, M., Huggins-Daines, D., Torkornoo, D.  ReadAlong Studio: Practical Zero-Shot Text-Speech Alignment for Indigenous Language Audiobooks. In proceedings of \textit{ SIGUL2022 @ LREC2022 }.  European Language Resources Association (ELRA).  
                                                                                                            \item[2020a] Brinklow, N., Littell, P., Lothian, D., Pine, A., Souter, H.  Indigenous Language Technologies \& Language Reclamation in Canada. In proceedings of \textit{ International Conference Language Technologies for All (LT4All): Enabling Linguistic Diversity and Multilingualism Worldwide }.  UNESCO.  
                                                                                \item[2020b]  Kuhn,  R.,  Davis,  F.,  D{\'e}silets,  A.,  Joanis,  E.,  Kazantseva,  A.,  Knowles,  R.,  Littell,  P.,  Lothian,  D.,  Pine,  A.,  Running Wolf,  C.,  Santos,  E.,  Stewart,  D.,  Boulianne,  G.,  Vishwa,  G.,  Maracle,  O.,  Martin,  A.,  Cox,  C.,  Junker,  M.,  Sammons,  O.,  Torkornoo,  D.,  Brinklow,  N.,  Child,  S.,  Farley,  B.,  Huggins-Daines,  D.,  Rosenblum,  D.,  Souter,  H.  The Indigenous Languages Technology Project at NRC Canada: an empowerment-oriented approach to developing language software. \textit{ Technical Report }.  National Research Council Canada.  
                                                                                                                                                                            \item[2018a]  Hall,  K.,  Pine,  A.,  Schwan,  M.  Doing Phonological Corpus Analysis in a Fieldwork Context. \textit{ Wa7 xweys{\'a}s i nqwaluttenh{\'\i}ha: He loves the peoples' languages. Essays in honour of Henry Davis }.  University of British Columbia.  
                                                                                \item[2018b] Pine, A., Turin, M.  Seeing the Heiltsuk Orthography from Font Encoding through to Unicode: A Case Study Using Convertextract. In proceedings of \textit{ CCURL 2018: The 3rd Workshop on Collaboration and Computing for Under-Resourced Languages }.  European Language Resources Association.  
                                                                                \item[2018c] Kazantseva, A., Maracle, B., Maracle, R., Pine, A.  Kawenn{\'o}n:nis: the Wordmaker for Kanyen'k{\'e}ha (Ohsweken Mohawk). In proceedings of \textit{ COLING: The 27th International Conference on Computational Linguistics }.  
                                                                                \item[2018d] Littell, P., Kazantseva, A., Kuhn, R., Pine, A., Arppe, A., Cox, C., Junker, M.  Indigenous language technologies in Canada: Assessment, challenges, and successes. In proceedings of \textit{ COLING: The 27th International Conference on Computational Linguistics }.  
                                                                                                            \item[2017a]  Pine,  A.,  Turin,  M.  Language Revitalization. \textit{ Oxford Research Encylopedia of Linguistics }.  Oxford University Press.  
                                                                                \item[2017b] Littell, P., Pine, A., Davis, H.  Waldayu and Waldayu Mobile: Modern digital dictionary interfaces for endangered languages. In proceedings of \textit{ ComputEL-2: The 2nd Workshop on Computational Methods for Endangered Languages }.  
                                                                                \item[2017c]  Forbes,  C.,  Davis,  H.,  Schwan,  M.,  Gitksan Research Lab,   Three Gitksan texts. \textit{ University of British Columbia Working Papers in Linguistics }.  University of British Columbia.  
                                                        \end{cvlist}

        \begin{cvlist}{Presentations \& Guest Lectures}
                    \item[2023]  Pine,  A.   \textit{{\textquotedblleft}Watch me Speak!{\textquotedblright} Interactive Storytelling using ReadAlong-Studio}. Linguistics Department. Virtual (University of California Santa Barbara).  
                    \item[2023]  Littell,  P. ,  Pine,  A.   \textit{Interactive Storytelling \& Practical Zero-Shot Text-Speech Alignment using ReadAlong-Studio}. Lanfrica Talk Series. Virtual (Masakhane).  
                    \item[2023]  Pine,  A.   \textit{Language Revitalization \& Speech Technology}. ASTU 402 Living Language: Science \& Society Guest Lecture. Virtual (University of British Columbia).  
                    \item[2023]  Pine,  A.   \textit{Language Revitalization \& Speech Technology}. Knowledge, Information and Technology Services (KITS). Virtual.  
                    \item[2022]  Pine,  A.   \textit{Speech Technology \& Language Reclamation}. FNEL 180. University of British Columbia.  
                    \item[2022]  Pine,  A.   \textit{Requirements and Motivations of Low-Resource Speech Synthesis for Language Revitalization}. Best Papers Award Session Plenary at ACL 2022. Dublin.  
                    \item[2022]  Pine,  A.   \textit{Gᵢ2Pᵢ: Rule-based, index-preserving grapheme-to-phoneme transformations}. ComputEL-5. Dublin.  
                    \item[2022]  Pine,  A.   \textit{Requirements and Motivations of Low-Resource Speech Synthesis for Language Revitalization}. Centre for Speech Technology Research. Edinburgh.  
                    \item[2022]  Pine,  A.   \textit{Developing Speech \& Language Technologies for Indigenous Language Revitalization}. Domain Adaptation Group. University of Toronto.  
                    \item[2022]  Pine,  A. ,  Huggins-Daines,  D. ,  Joanis,  E. ,  Littell,  P. ,  Tessier,  M. ,  Torkornoo,  D.   \textit{{\textquotedblleft}Watch me Speak!{\textquotedblright} Interactive Storytelling using ReadAlong-Studio}. Community Workshop for Indigenous Language Technology (CWILTs Series). Virtual (National Research Council).  
                    \item[2022]  Pine,  A. ,  Huggins-Daines,  D. ,  Joanis,  E. ,  Littell,  P. ,  Tessier,  M. ,  Torkornoo,  D.   \textit{{\textquotedblleft}Watch me Speak!{\textquotedblright} Interactive Storytelling using ReadAlong-Studio}. ICLDC 2023: Centering Justice in Language Work. Virtual (University of Hawai'i).  
                    \item[2021]  Pine,  A.   \textit{Low Resource Speech Synthesis}. NLP Seminar. National Research Council Canada.  
                    \item[2020]  Huggins-Daines,  D. ,  Littell,  P. ,  Pine,  A. ,  Joanis,  E.   \textit{Readalongs: Automatic alignment of speech and text for Indigenous language audiobooks}. 52nd Algonquian Conference. University of Wisconsin - Madison.  
                    \item[2020]  Pine,  A.   \textit{Language Technology \& Language Reclamation}. FNEL 180. University of British Columbia.  
                    \item[2020]  Pine,  A.   \textit{Lexicography, Language Revitalization \& Mother Tongues Dictionaries}. FNEL 382. University of British Columbia.  
                    \item[2020]  Pine,  A.   \textit{Language Revitalization \& Technology}. ASTU 402. University of British Columbia.  
                    \item[2020]  Pine,  A.   \textit{G2P (grapheme-to-phoneme) the 'what', 'so what', and 'now what'}. Continual Workshops in Indigenous Language Technology (CWILTs). Virtual.  
                    \item[2019]  Pine,  A. ,  Turin,  M.   \textit{How to Make a 'Mother Tongues' Digital Dictionary}. ICLDC 6. University of Hawaii.  
                    \item[2019]  Pine,  A.   \textit{Digital Kinesthetic Learning and the Gitksan Pronominal Paradigm}. ICLDC 6. University of Hawaii.  
                    \item[2019]  Kazantseva,  A. ,  Brant,  R. ,  Maracle,  O. ,  Maracle,  R. ,  Pine,  A.   \textit{Kawenn{\'o}n:nis: An Online Tool for Learning Verbal Morphology in Kanyen'k{\'e}ha}. ICLDC 6. University of Hawaii.  
                    \item[2019]  Pine,  A. ,  Littell,  P. ,  Maracle,  R.   \textit{Mohawk O'Clock: Limited Domain Speech Synthesis in Kanyen'k{\'e}ha}. ICLDC 6. University of Hawaii.  
                    \item[2019]  Pine,  A.   \textit{UBC Alumni Panel}. 50th Anniversay of the UBC Linguistics Department. University of British Columbia.  
                    \item[2019]  Pine,  A.   \textit{Language Revitalization \& the Role of Technology}. Nerd Nite YYJ. Victoria Event Centre.  
                    \item[2019]  Pine,  A.   \textit{Language Revitalization, technology, and making the most of your undergraduate experience}. FNEL 180. University of British Columbia.  
                    \item[2019]  Pine,  A.   \textit{Language Revitalization \& Technology}. ASTU 402 - Living Language: Science \& Society. University of British Columbia.  
                    \item[2019]  Pine,  A.   \textit{Kawenn{\'o}n:nis the Word Maker for Kanyen'k{\'e}ha}. Kanyen'k{\'e}ha Mohawk 101R. Renison University College at University of Waterloo.  
                    \item[2019]  Pine,  A.   \textit{Kawenn{\'o}n:nis the Word Maker for Kanyen'k{\'e}ha}. Kanyenn'k{\'e}ha Mohawk Class. Tsi Ty{\'o}nnheht Onkwaw{\'e}n:na, Tyendinaga.  
                    \item[2019]  Pine,  A. ,  Kazantseva,  A.   \textit{Supplementing text-based Indigenous language applications with synthetic speech}. Ottawa AI. NRC Sussex, Ottawa.  
                    \item[2019]  Daines,  D. ,  Littell,  P. ,  Pine,  A. ,  Joanis,  E. ,  Torkornoo,  D.   \textit{ReadAlong Studio: Zero-shot Text/Speech Alignment for Indigenous Language Audiobooks}. Ottawa AI. NRC Sussex, Ottawa.  
                    \item[2019]  Pine,  A. ,  Brinklow,  N. ,  Souter,  H. ,  Lothian,  D.   \textit{Canadian National Research Council's Indigenous	Language Technology Project}. International Conference on Language Technologies for All. UNESCO HQ, Paris.  
                    \item[2019]  Pine,  A.   \textit{Developing technology to support Indigenous language revitalization in Canada}. Centre for Speech Technology Research (CSTR). University of Edinburgh.  
                    \item[2018]  Pine,  A.   \textit{Practical Phonetics}. Adult Gitxsan Class. Gitsegukla, \& Terrace, B.C.  
                    \item[2018]  Pine,  A.   \textit{Digital Kinesthetic Learning and the Gitksan Pronominal Paradigm}. Research in Indigenous Languages and Linguistics. University of Victoria.  
                    \item[2018]  Pine,  A.   \textit{An Introduction to Gitksan Research}. LING 372 Guest Lecture. University of Victoria.  
                    \item[2018]  Pine,  A.   \textit{A multi-dialectal Ayajuthem talking dictionary}. . Sliammon, B.C.  
                    \item[2018]  Pine,  A.   \textit{Ama silkw sa}. Gitksan Elders Meeting. Gitksan Elder's Center. Gitanmaax, B.C.  
                    \item[2018]  Pine,  A.   \textit{Seeing the Heiltsuk Orthography from Font Encoding to Unicode: A Case Study Using Convertextract}. Conference on Collaboration and Computing of Under Resourced Languages. Miyazaki, Japan.  
                    \item[2018]  Kazantseva,  A. ,  Pine,  A.   \textit{Kawenn{\'o}n:nis: The Wordmaker for Kany{\'e}n'keha}. Onkwawenna Kentyohkwa Immersion School. Ohsweken, Ontario.  
                    \item[2018]  Pine,  A. ,  Turin,  M.   \textit{Developing Digital Tools for Language Revitalization: Demystifying Coding, Apps, and Web Platforms}. CoLang. University of Florida.  
                    \item[2018]  Pine,  A. ,  Turin,  M.   \textit{Developing Digital Tools for Language Revitalization: Demystifying Coding, Apps, and Web Platforms}. National Institute for Japanese Language and Linguistics. Tokyo, Japan.  
                    \item[2018]  Pine,  A.   \textit{An Introduction to Gitksan Research}. LING 372 Guest Lecture. University of Victoria.  
                    \item[2017]  Pine,  A.   \textit{Introducing Waldayu \& Waldayu Mobile}. ComputEL-2: The 2nd Workshop on Computational Methods for Endangered Languages. University of Hawaii.  
                    \item[2017]  Pine,  A.   \textit{Mobile App Development}. BC Breath of Life Institute. University of British Columbia.  
                    \item[2017]  Pine,  A.   \textit{Lexicography for Endangered Languages}. FNEL 382 Guest Lecture. University of British Columbia.  
                    \item[2017]  Smith,  J. ,  Pine,  A.   \textit{Dim liseewis Xsiwis g̲ant Sginist wil wihl hahlaalst sit'aadiit}. Gitksan Language Symposium. Gitksan Wet'suwet'en Education Society. Hazelton, B.C..  
                    \item[2017]  Pine,  A.   \textit{Ama silkw sa}. Gitksan Elders Meeting. Gitksan Elder's Center. Gitanmaax, B.C.  
                    \item[2017]  Pine,  A.   \textit{Sim ayeehl k'uuhl}. FNEL 180 Guest Lecture. University of British Columbia.  
                    \item[2017]  Pine,  A.   \textit{A visual presentation with Aidan Pine: The possibilities of our new app}. Ayajuthem language conference. Sliammon, B.C.  
                    \item[2017]  Pine,  A.   \textit{Practical Phonetics}. Adult Gitxsan Class. Gitanmaax, B.C.  
                    \item[2016]  Pine,  A.   \textit{Mobilizing Language Data: An Endangered Language Dictionary App}. Multidisciplinary Undergraduate Research Conference. University of British Columbia.  
                    \item[2016]  Pine,  A.   \textit{Making the most of undergraduate research}. FNEL 180 Guest Lecture. University of British Columbia.  
                    \item[2015]  Pine,  A.   \textit{Connecting Linguistics to Language Learners: a Lesson Plan for Gitksan}. Multidisciplinary Undergraduate Research Conference. University of British Columbia.  
                    \item[2015]  Pine,  A.   \textit{'Nit gwihl Jabisi'm}. FNEL 180 Guest Lecture. University of British Columbia.  
                    \item[2014]  Barois,  M. ,  Bicevskis,  K. ,  Cheng,  K. ,  Pine,  A.   \textit{Quantifier Scope in Gitksan}. Gitksan Research Group Research Seminar. University of British Columbia.  
                    \item[2014]  Pine,  A.   \textit{Preferential Encliticization in Gitksan}. Linguistics Department. University of British Columbia.  
                   
        \end{cvlist}
        \begin{cvlist}{Other Languages}
                            \item \textbf{Danish} - Fluent. Pr{\o}ve i Dansk 3 Fluency Certificate
                            \item \textbf{French} - Fluent. Double Dogwood High School Diploma in English \& French
                            \item \textbf{Gitksan} - Reading/Writing: literate; Oral: Intermediate. 
                            \item \textbf{h{\textschwa}n̓q̓{\textschwa}min̓{\textschwa}m} - Intermediate. UBC FNLG 101, 102, 201, 202, \& 301
                    \end{cvlist}
        
         \begin{cvlist}{Skills}
                        \item Mobile development with  \textit{Ionic},  \textit{TypeScript},  \textit{Angular},  \textit{Sass}                         \item Web development with  \textit{HTML},  \textit{CSS/Sass},  \textit{Angular},  \textit{Flask},  \textit{Stencil}                         \item API development with  \textit{Flask},  \textit{Fastapi}                         \item Programming with  \textit{Python},  \textit{Racket/Scheme},  \textit{iPython},  \textit{Pytorch}                         \item Acoustic Analysis/Editing with  \textit{Praat},  \textit{Wavesurfer},  \textit{Audacity},  \textit{ffmpeg}                         \item Speech Synthesis with  \textit{Festival},  \textit{FastSpeech2},  \textit{Tacotron2}                         \item Typesetting/Graphic design with  \textit{Photoshop},  \textit{InDesign},  \textit{\LaTeX},  \textit{Balsamiq}                      \end{cvlist}

         \begin{cvlist}{Employment}
                            \item[Jan 2018 -  Present ] \textbf{Application Development Specialist - National Research Council Canada}
                \newline Developer and researcher for NRC Indigenous Language Technology project
                            \item[Oct 2016 -  Jun 2019 ] \textbf{Software Developer - University of British Columbia}
                \newline Lead programmer for the development of four online (web and mobile) dictionaries.
                            \item[Oct 2016 -  Present ] \textbf{Software Developer - First Peoples' Heritage and Language Council of Canada}
                \newline Lead developer for redesigning, engineering and publishing 15 of FPHLCC's mobile dictionary applications for Android and iOS. Independent contract through Mother Tongues
                            \item[Aug 2016 -  Jan 2017 ] \textbf{Software Developer - Simon Fraser University}
                \newline Lead programmer for the development of a number of orthography conversion tools fo Heiltsuk and Tsilhqot'in languages. Independent contract through Mother Tongues
                            \item[May 2016 -  Present ] \textbf{Web Developer - Self Employed}
                \newline Freelance website design and development
                            \item[Jan 2016 -  Aug 2017 ] \textbf{Translator and Computational Linguist - SmartKYC (Remote Work)}
                \newline Natural language processing, writing regular expressions to capture inflectional morphological rules in French and Danish. Independent contract
                            \item[Jan 2015 -  Jun 2019 ] \textbf{Technology consultant - University of British Columbia}
                \newline Consulting and programming for the First Nations and Endangered Language program.
                            \item[May 2013 -  Sep 2019 ] \textbf{Juma Food Truck Owner \& Operator - Self Employed}
                \newline I designed, commissioned to build, own and operate a food truck that runs seasonally on Vancouver Island and employs full time and part time staff
                            \item[Apr 2013 -  Jan 2014 ] \textbf{Copy Editor \& Research Assistant - University of British Columbia}
                \newline Assisted in editing interlinear glossing and linguistics-related texts in the book: Written as I Remember It: Teachings ({\textglotstop}{\textschwa}ms ta{\textglotstop}aw) from the Life of a Sliammon Elder.
                    \end{cvlist}
        
        \begin{cvlist}{Volunteer}
                            \item[June 2017 - present] \textbf{Hesquiaht Dictionary}
                \newline Help with data processing, web and mobile app development for the Hesquiaht language dictionary
                            \item[May 2016 - September 2020] \textbf{Squamish Dictionary}
                \newline Help with publishing online 'talking dictionary' for Squamish language in partnership with Peter Jacobs
                            \item[Jan 2016 - May 2016] \textbf{Python for Lexicographers Workshop Series}
                \newline Teach basic Python and programming skills to undergraduate, graduate and post-graduate lexicographers at the UBC linguistics department
                            \item[2014 - 2016] \textbf{x{\textsuperscript{w}}na{\textglotstop}{\textschwa}l{\textschwa}mx{\textsuperscript{w}} s{\ensuremath{\chi}}{\textschwa}{\ensuremath{\chi}}i:ls Journal Review}
                \newline UBC Indigenous Studies Submission Reviewer
                            \item[2013 - 2016] \textbf{Doreen Jensen Memorial Gitksan Class}
                \newline Assist students by explaining grammatical, phonological and orthographic concepts
                    \end{cvlist}

        \begin{cvlist}{Awards \& Grants}
                        \item[2024] National Research Council Digital Technology 'Value for Canada' Award
                        \item[2022] NRC IP Achievement Award (as part of Indigenous Language Technology Project)
                        \item[2022] ACL2022 Best Special Theme Paper
                        \item[2022] National Research Council Digital Technology 'Research \& Technology Breakthrough' Award
                        \item[2020] Scotland's Saltire Scholarship
                        \item[2018] National Research Council Digital Technology 'Value for Canada' Award
                        \item[2016] Sanderson Family Service Award
                        \item[2016] 1st Place Oral Presentation at UBC Multidisciplinary Undergraduate Research Conference
                        \item[2016] Dean's Outstanding Leadership Award for Academic Contribution to the Faculty of Arts
                        \item[2015] UBC Arts Undergraduate Research Award
                        \item[2015] UBC Student Scholarship in Arts
                        \item[2014] UBC Arts Undergraduate Research Award
                    \end{cvlist}

    \end{cv}
    
    \end{document}